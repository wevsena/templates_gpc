\documentclass[$if(fontsize)$$fontsize$,$endif$$if(lang)$$lang$,$endif$$if(papersize)$$papersize$,$endif$$for(classoption)$$classoption$$sep$,$endfor$]{$documentclass$}
$if(fontfamily)$
\usepackage{$fontfamily$}
$else$
%\usepackage{lmodern}
\usepackage[sfdefault,condensed]{roboto}
$endif$
$if(linestretch)$
\usepackage{setspace}
\setstretch{$linestretch$}
$endif$
\usepackage{amssymb,amsmath}
\usepackage{ifxetex,ifluatex}
\usepackage{fixltx2e} % provides \textsubscript
\ifnum 0\ifxetex 1\fi\ifluatex 1\fi=0 % if pdftex
  \usepackage[T1]{fontenc}
  \usepackage[utf8]{inputenc}
$if(euro)$
  \usepackage{eurosym}
$endif$
\else % if luatex or xelatex
  \ifxetex
    \usepackage{mathspec}
    \usepackage{xltxtra,xunicode}
  \else
    \usepackage{fontspec}
  \fi
  \defaultfontfeatures{Mapping=tex-text,Scale=MatchLowercase}
  \newcommand{\euro}{€}
$if(mainfont)$
    \setmainfont{$mainfont$}
$endif$
$if(sansfont)$
    \setsansfont{$sansfont$}
$endif$
$if(monofont)$
    \setmonofont[Mapping=tex-ansi]{$monofont$}
$endif$
$if(mathfont)$
    \setmathfont(Digits,Latin,Greek){$mathfont$}
$endif$
\fi
% use upquote if available, for straight quotes in verbatim environments
\IfFileExists{upquote.sty}{\usepackage{upquote}}{}
% use microtype if available
\IfFileExists{microtype.sty}{%
\usepackage{microtype}
\UseMicrotypeSet[protrusion]{basicmath} % disable protrusion for tt fonts
}{}
$if(geometry)$
\usepackage[$for(geometry)$$geometry$$sep$,$endfor$]{geometry}
$endif$
$if(lang)$
\ifxetex
  \usepackage{polyglossia}
  \setmainlanguage{$mainlang$}
\else
  \usepackage[shorthands=off,$lang$]{babel}
\fi
$endif$
$if(natbib)$
\usepackage{natbib}
\bibliographystyle{$if(biblio-style)$$biblio-style$$else$plainnat$endif$}
$endif$
$if(biblatex)$
\usepackage{biblatex}
$if(biblio-files)$
\bibliography{$biblio-files$}
$endif$
$endif$
$if(listings)$
\usepackage{listings}
$endif$
$if(lhs)$
\lstnewenvironment{code}{\lstset{language=Haskell,basicstyle=\small\ttfamily}}{}
$endif$
$if(highlighting-macros)$
$highlighting-macros$
$endif$
$if(verbatim-in-note)$
\usepackage{fancyvrb}
\VerbatimFootnotes
$endif$
$if(tables)$
\usepackage{longtable,booktabs}
$endif$
$if(graphics)$
\usepackage{graphicx}
\makeatletter
\def\maxwidth{\ifdim\Gin@nat@width>\linewidth\linewidth\else\Gin@nat@width\fi}
\def\maxheight{\ifdim\Gin@nat@height>\textheight\textheight\else\Gin@nat@height\fi}
\makeatother
% Scale images if necessary, so that they will not overflow the page
% margins by default, and it is still possible to overwrite the defaults
% using explicit options in \includegraphics[width, height, ...]{}
\setkeys{Gin}{width=\maxwidth,height=\maxheight,keepaspectratio}
$endif$
\ifxetex
  \usepackage[setpagesize=false, % page size defined by xetex
              unicode=false, % unicode breaks when used with xetex
              xetex]{hyperref}
\else
  \usepackage[unicode=true]{hyperref}
\fi
\hypersetup{breaklinks=true,
            bookmarks=true,
            pdfauthor={$author-meta$},
            pdftitle={$title-meta$},
            colorlinks=true,
            citecolor=$if(citecolor)$$citecolor$$else$blue$endif$,
            urlcolor=$if(urlcolor)$$urlcolor$$else$blue$endif$,
            linkcolor=$if(linkcolor)$$linkcolor$$else$magenta$endif$,
            pdfborder={0 0 0}}
\urlstyle{same}  % don't use monospace font for urls
$if(links-as-notes)$
% Make links footnotes instead of hotlinks:
\renewcommand{\href}[2]{#2\footnote{\url{#1}}}
$endif$
$if(strikeout)$
\usepackage[normalem]{ulem}
% avoid problems with \sout in headers with hyperref:
\pdfstringdefDisableCommands{\renewcommand{\sout}{}}
$endif$
\setlength{\parindent}{0pt}
\setlength{\parskip}{6pt plus 2pt minus 1pt}
\setlength{\emergencystretch}{3em}  % prevent overfull lines
$if(numbersections)$
\setcounter{secnumdepth}{5}
$else$
\setcounter{secnumdepth}{0}
$endif$
$if(verbatim-in-note)$
\VerbatimFootnotes % allows verbatim text in footnotes
$endif$

%%% Use protect on footnotes to avoid problems with footnotes in titles
\let\rmarkdownfootnote\footnote%
\def\footnote{\protect\rmarkdownfootnote}

$if(compact-title)$
%%% Change title format to be more compact
\usepackage{titling}

% Create subtitle command for use in maketitle
\newcommand{\subtitle}[1]{
  \posttitle{
    \begin{center}\large#1\end{center}
    }
}

\setlength{\droptitle}{-2em}
$endif$

% \title{
% \noindent
% \large\textbf{$title$} \hfill \textbf{$first-name$ $last-name$} \\
% \normalsize Module: $module$ \hfill Team members: $for(teammates)$$teammates$$sep$, $endfor$ \\
% Module Lead: $module-lead$ \hfill Data do laboratório: $lab-date$ \\
% Lab Lead: $TA$ \hfill Due Date: $due-date$
% }
% \date{\vspace{-5ex}}

$for(header-includes)$
$header-includes$
$endfor$

%\documentclass[a4paper, 11pt, brazil]{article}
\usepackage[brazilian]{babel}
\usepackage{indentfirst}
\usepackage{scrextend}
\usepackage[utf8]{inputenc}
\usepackage{lipsum}
% \usepackage{fontspec}
% \usepackage[default,oldstyle,scale=0.95]{opensans}
\usepackage{geometry}
\usepackage{fancyhdr}
\usepackage{graphicx}
\usepackage[fit]{truncate}
\geometry{
	a4paper,
	total={170mm,257mm},
	left=25mm,
	top=25mm,
	right=20mm,
	bottom=20mm
}
\usepackage{pdfpages}
\renewcommand{\baselinestretch}{1.25}

\pagestyle{fancy}
\rfoot{\thepage}
\cfoot{}

%VERSÃO DE TESTES---------------------------------------------------------------
$if(provisorio)$
\usepackage[text=Provisório, scale=0.7,color={[gray]{0.85}}]{draftwatermark}
$endif$


\begin{document}


\includepdf[pages=-]{capa_tematica.pdf}


%Capa---------------------------------------------------------------------------


\thispagestyle{plain}




\begin{figure}[t]
\includegraphics[scale=0.75]{barra_tce.png}
\centering
\end{figure}

\begin{center}
	{\Large \textbf{TRIBUNAL DE CONTAS DO ESTADO DA PARAÍBA}} \\
	{\large Diretoria de Auditoria e Fiscalização - DIAFI} \\
	%{\large Grupo de Planejamento e Controle - GPC} \\
\end{center}

\thispagestyle{empty}

\vspace{4cm}

\begin{center}
{\Large \textbf{$numerorel$}}

\end{center}


\vspace{1cm}

% Assunto do relatório - capa---------------------------------------------------
\begin{center}
	{\Huge \textbf{$assunto$}} \\
	\vspace{0.5cm}
	{\Large $subassunto$}

\end{center}

\vspace{10cm}

\begin{center}
	$mes_ano$
\end{center}



\newpage

\thispagestyle{empty}


% Contra-capa-------------------------------------------------------------------

\vspace*{\fill}
\begin{center}
	{\large \textbf{Nome do Presidente do TCE-PB}} \\
	{\normalsize {Presidente do TCE-PB}} \\
	\vspace{1cm}
	{\large \textbf{Nome do Diretor da DIAFI}} \\
	{\normalsize {Diretor da DIAFI}} \\
	\vspace{1cm}
	{\large \textbf{Organização}} \\
	{\normalsize {Organizador 1}} \\
	{\normalsize {Organizador 2}} \\
	%{\normalsize {Grupo de Auditoria Operacional - GAOP}} \\
	%{\normalsize {Núcleo de Avaliação e Engenharia - NAVE}} \\
	\vspace{1cm}
	{\large \textbf{Colaboração e participação}} \\
	{\normalsize {Departamento 1}} \\
	{\normalsize {Departamento 2}} \\
	{\normalsize {Departamento 3}} \\
	{\normalsize {Departamento 4}} \\
	{\normalsize {Departamento 5}} \\
	\vspace{1cm}
	{\large \textbf{Imagem da capa}} \\
	{\normalsize {Nome do autor da imagem da capa}} \\
		\vspace{1cm}
\end{center}

\vspace*{\fill}

\newpage


%Tabela de conteúdos------------------------------------------------------------
$if(toc)$
{
\thispagestyle{plain}
\hypersetup{linkcolor=black}
\setcounter{tocdepth}{$toc-depth$}
\tableofcontents
}
$endif$
$if(lot)${
\hypersetup{linkcolor=black}
\listoftables
}
$endif$
$if(lof)$
\hypersetup{linkcolor=black}
\listoffigures
$endif$

\newpage

\setlength{\parindent}{4em}
\setlength{\parskip}{1em}


%Fancy--------------------------------------------------------------------------

\fancyhead[L]{\truncate{0.45\headwidth}\rightmark}
\fancyhead[R]{\truncate{0.5\headwidth}\leftmark}

%-------------------------------------------------------------------------------


$body$

\end{document}

